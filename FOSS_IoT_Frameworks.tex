\documentclass{article}
\usepackage[utf8]{inputenc}

\title{FOSS IoT Frameworks review}
\author{Benoit RENAULT}
\date{2017/08/10}

%% Margins
\usepackage[
    top=3cm,
    bottom=3cm,
    left=3cm,
    right=3cm
]{geometry}

\usepackage{natbib}
\usepackage{graphicx}
\usepackage{hyperref}

\begin{document}

\maketitle

\newpage

\tableofcontents

\newpage

\section{Introduction}

The context for this document is the need for an IoT server Gateway component for the \href{https://github.com/occiware/occiware-ozwillo}{OCCIware LinkedData Use Case}. As the OCCIware project is FOSS\footnote{Free or Open-Source Software}, the component \textbf{must} be open-source itself. Therefore, what you will find here is a short state of the art of such frameworks, and no consideration shall be given to proprietary solutions.

\emph{Please note that, while the evaluation criteria may be considered timeless, the analyses presented below only reflect the state of the different projects as of this document's last update date, written on the first page.}

\section{Evaluation criteria}

Each of the criteria is presented below with at least one argument as to how it allows to measure a given characteristic. The criteria are also chosen so that each solution needs not be analyzed more than one hour. For qualitative criteria, a little argumentation will be given in the detailed analysis (noted with excellent/good/average/poor/non-existent). Rational critics are, of course, welcome.

Some of them come from \href{https://www.thingworx.com/wp-content/uploads/WP_oreilly-media_evaluating-and-choosing-an-iot-platform_978-1-491-95203-0_EN.pdf}{}, and the ones about security have been inspired by \href{https://www.owasp.org/index.php/IoT_Framework_Assessment}{Craig SMITH's "IoT Framework Assessment" OWASP page}.

\subsection{General criteria}

Here are some general criteria that allow us to assess the quality of any FOSS project.

\paragraph{Project Health} Determines how "mature" the project is, and its potential to last on the long term.

\begin{itemize}
\item First Release Date: The older, the better, might clue a position of pionneer, or a certain stability.
\item Latest Release Date: The newer, the better, might clue that the project's release cycle is still active.
\item Latest Commit Date: The newer, the better, if the project's release cycle is slow, might clue that the project is still ongoing active development.
\item Number of main contributors: The higher, the better. If equal to one, exercise caution. Main contributors have proportional contributions to the code repository.
\item Open issues ratio: The lower, the better. Is equal to the number of open issues over the total number of issues. Might clue at a good responsivity to user input, be it for bugs or improvements.
\end{itemize}

\paragraph{Company backing} Partly shows how reliable the software can be: if there is a company ready to back it up and provide support to customers, and if there are other companies using the software, it might mean that it is trustworthy.

\begin{itemize}
\item Company Support: The better the company support is, the better the appreciation.
\item Company Adoption: The more the software is adopted by other actors, the better the appreciation.
\end{itemize}

\paragraph{Documentation} A well-documented project is a must.

\begin{itemize}
\item Currentness: The documentation must be up to date with the code. The more it respects this principle, the better the appreciation.
\item Adaptability: Documentation must be adapted to the different users. The more it respects this principle, the better the appreciation.
\end{itemize}

\paragraph{UI-UX} The software must provide appropriate tools at all levels, be it CLIs or Graphical UIs, depending on the context.

\begin{itemize}
\item Server Management UI: the more powerful and concise, the better the appreciation.
\item Sample applications: the more there are and the more complete, the better the appreciation.
\end{itemize}

\subsection{IoT-specific criteria}

Here are some general criteria that allow us to assess the quality of IoT projects in particular.

\paragraph{Security} A key characteristic for IoT systems. As we cannot conduct a security assessment ourselves, we will rather look for a commitment to security and whether audits have been made or not. Security must be ensured at all levels: connectivity, storage, update, authentication, appropriate management of devices...

\begin{itemize}
\item Statements: the more is explained on the security measures taken in the documentation, the better the appreciation. Clues at a concern and understanding of the IoT problematics.
\item Audits: the more positive reviews about the security are, the better the appreciation. Clues at a good applications of best security practices.
\end{itemize}

\paragraph{Connectivity/Flexibility} Is very important for IoT systems, because they are supposed to connect many elements that are all very different.

\begin{itemize}
\item Number of compatible hardware platforms: The higher, the better. Might clue to a shorter time-to-market.
\item Number of supported protocols: The higher, the better. Might clue at a will not to lock the user in the system.
\item Number of SDK implementations: The higher, the better. Might clue at an easier time for in-house developpers to incorporate the solution with their own prefered language.
\item Modularity: the more the components of the solution can easily be changed (for instance, the DB), the better the appreciation.
\end{itemize}

\paragraph{Scalability} As IoT systems tend to grow really fast, it is necessary that the application scales by design, and allows for many devices, all over the world, to connect to it, without failure.

\begin{itemize}
\item Third-Party scalability: IoT solutions are often a combination of many pre-existing FOSS projects. This criteria evaluates whether the 3rd-Party solutions chosen by the IoT platform are capable of scaling: the more they are, the better the appreciation.
\item Platform scalability: IoT solutions must also be scalable themselves by being capable of being distributed across different machines and still be able to talk to one another smoothly. This criteria evaluates whether the platform is capable of scaling: the more it id, the better the appreciation.
\end{itemize}

\subsection{Application-specific criteria}

Here are some specific criteria that are related to the peculiar needs of the OCCIware project.

\paragraph{Arduino/NodeMCU compatibility} Since it is the most popular platform for hardware experimentation, it is an important criteria that the service musts provide good support for it.

\begin{itemize}
\item Library: The more complete and recognized the library, the better the appreciation.
\item Boilerplate code: The smaller the boilerplate code, the better the appreciation.
\end{itemize}

\paragraph{Java} The framework must be based on Java for easy editing of the sources and better maintainability, since it is the main language of the author.

\begin{itemize}
\item Server percentage: The higher the quantity of Java code for the server, the better.
\end{itemize}

\section{Detailed analysis}

\subsection{SiteWhere}

\paragraph{Project Health}

\begin{itemize}
\item First Release Date: The development started in 2010 as a closed source, asset tracking platform. The first open source release happened on \textbf{2014/03/06}. With 7 years of age, it can be assumed it is quite mature and stable.
\item Latest Release Date: \textbf{2017/06/19} - Still releases new versions as of today.
\item Latest Commit Date: \textbf{2017/06/20} - Work is still ongoing to deliver a new version.
\item Number of main contributors: \textbf{1} - It seems to be mainly a one-man project, which is a problem. Some other people have contributed little bits of code, but not to the extent the main contributor has.
\item Open issues ratio: \textbf{0.0534653465} - With 505 issues opened by many various actors (and not just the main contributor as milestones) since the project's arrival on Github, only 27 remain today, which is a very, very good. Excellent reactivity.
\end{itemize}

Good project health overall, though a shame that it seems to have only one main developer behind it.

\paragraph{Company backing}

\begin{itemize}
\item Company Support: \textbf{excellent} - the project is backed by a well-established company (seven years of existence), and they offer a separate enterprise version with more features. Support is available to companies through the possibility to buy block hours for assistance, and also to anyone using the community edition through a rather active Google Group instance.
\item Company Adoption: \textbf{good} - the corporate page of the project gives 2 comments by supposed customers, no success story using sitewhere was found but there were a few press articles talking about it like \href{https://opensourceforu.com/2017/07/sitewhere-open-platform-connected-devices/?utm_content=buffer4c828&utm_medium=social&utm_source=twitter.com&utm_campaign=buffer}{this one}. It can however be infered that if the company has existed for 7 years, it means that they have found clients that appreciate their services.
\end{itemize}

Good company backing overall, though it's a shame they don't communicate a bit more about their customers.

\paragraph{Documentation}

\begin{itemize}
\item Currentness: \textbf{good} - From the short time spent in the documentation, it seems it has been updated to match the latest release (as its number is written on the documentation homepage), however, in the absence of dating on all pages, it makes it more difficult to say if everything has been updated or not : for that, you have to go to the \href{https://github.com/sitewhere/sitewhere-documentation/}{documentation Github Repository}.
\item Adaptability: \textbf{excellent} - The documentation is well separated between user and developper use cases. Explanations are clear, with lots of screenshots and appropriate commands where needed. The README.md on the Github Repository is crystal clear. Different levels of documentation are provided, like architecture, technologies, usage, code structure, ...
\end{itemize}

Good documentation overall, does make you want to use the product.

\paragraph{UI-UX}

\begin{itemize}
\item Server Management UI: \textbf{good} - Powerful and well-documented web administration interface, though its design is a little bit old school.
\item Sample applications: \textbf{average} - There are a few example applications for ios, android and webapps but not so many examples for hardware.
\end{itemize}

The proposed UIs seem good overall, but it would take a deeper examination to fully conclude on their capacity.

\paragraph{Security} 

\begin{itemize}
\item Statements: \textbf{average} - There is no open commitment to a secure system, on none of their websites. No explanation of the security measures they have taken, except that they use the Spring Frameworks integrated capabilities (which is good). No mention about the cryptographic means they use either.
\item Audits: \textbf{non-existent} - No mention on the product website, neither found any through a quick search on the web.
\end{itemize}

There is no apparent concern about security on their websites, which is certainly not good. Furthermore, there is apparently an SSL configuration problem with their corporate website, which is not a good sign.

\paragraph{Connectivity/Flexibility}

\begin{itemize}
\item Number of compatible hardware platforms: \textbf{4} - Android, Arduino, Raspberry Pi, and Generic Java-capable board (Cf. \href{http://documentation.sitewhere.io/integration.html}{dedicated documentation page}). Portentially highly compatible with many other platforms, but since no detail is given, we may assume with some reserves that not so many have actually been tested.
\item Number of supported protocols: \textbf{8} - MQTT, AMQP, OpenWire, XMPP, HTTP REST requests, WebSocket, Hazelcast, Stomp (Cf. \href{http://documentation.sitewhere.io/userguide/tenant/device-communication.html}{dedicated documentation page}).
\item Number of SDK implementations: \textbf{2} - Android and IOS.
\item Modularity: \textbf{good} - Design built out of preexisting opensource components. DB apparently easily changeable between MongoDB and Apache HBase (Cf. \href{http://documentation.sitewhere.io/architecture.html}{dedicated documentation page})
\end{itemize}

Overall, many connectors available, though seem to be a little weak on the hardware side, and lack of a SdK for Desktop applications.

\paragraph{Scalability}

\begin{itemize}
\item Third-Party scalability: \textbf{excellent} - Sitewhere offers a choice between MongoDB, Apache HBase and InfluxDB, which are three highly scalable Data Storage Technologies.
\item Platform scalability: \textbf{non-existent} - It seems a Sitewhere instance has no way to communicate with another and allow load-balancing.
\end{itemize}

It seems that Sitewhere's scalability claim is only based off the capabilities of its 3rd-parties components.

\paragraph{Arduino/NodeMCU compatibility}

\begin{itemize}
\item Library: \textbf{good} - Quite a lot of work seems to have been put into arduino compatibility, and the library seems very complete (with several example sketches included). However, no particular documentation about the NodeMCU/ESP8266.
\item Boilerplate code: \textbf{good} - The event/publish/subscribe structure, while being very efficient, requires here quite a lot of boilerplate code just to send a little bit of data.
\end{itemize}

Quite good Arduino compatibility, but it would really be welcome to have some documentation on its usage with NodeMCU/ESP8266 specifically.

\paragraph{Java} 

\begin{itemize}
\item Server percentage: \textbf{81.8\%} - The server is mainly just plain Java. Exactly what we need.
\end{itemize}

\subsection{Kaa}

\paragraph{Project Health}

\begin{itemize}
\item First Release Date: \textbf{2014/06/30} - The Kaa Project seems to have been open-sourced from its very beginning.
\item Latest Release Date: \textbf{2016/10/28} - The latest release dates back to the previous year, and could have us think that the project's activity has significantly dropped.
\item Latest Commit Date: \textbf{2017/06/31} - Actually, it seems that the project is still very much active, and a \href{https://www.youtube.com/watch?v=PaRSwYIGMG4}{recently posted video} on Youtube shows that they plan to release the 1.0 version during this summer. They seem to have reoriented all of their efforts into documenting/testing the project for the upcoming release.
\item Number of main contributors: \textbf{10} - There is a reasonable number of main contributors, whose work is rather well-distributed along time. All in all, the project has received contributions from 57 different contributors, which shows a good interest from the IoT community.
\item Open issues ratio: \textbf{0.0173032153} - A very low ratio, almost equal to zero : over the course of the project, 15893 issues have been opened, and now only 275 remain. It certainly goes to show that they have a proactive behaviour toward bugs. It is to be noted that they don't use Github Issues, but \href{http://jira.kaaproject.org/projects/KAA/issues/}{their own Jira instance}.
\end{itemize}

Kaa has a very good project health overall, especially since it has reached a point where the team behind it can trustfully say that they are ready for the 1.0 release.

\paragraph{Company backing}

\begin{itemize}
\item Company Support: \textbf{good} - The company behind Kaa, \href{https://www.kaaiot.io/}{KaaIoT}, seems to be more committed to corporate rather than community support (\href{https://stackoverflow.com/questions/44330714/why-are-the-questions-about-kaa-basically-not-being-answered}{see this StackOverflow discussion}), especially with the upcoming release.
\item Company Adoption: \textbf{good} - Given that the company has existed for 3 years, has, acccording to its LinkedIn page, between 51 and 200 employees, and that \href{https://www.kaaiot.io/company/careers/}{they are still recruiting a lot}, we can assume they have definitely found clients to buy one of the many formulas they offer on their corporate website. However, no trace was found on who these clients are.
\end{itemize}

Kaa is supported by a rather big team, and after 3 years of continued existence, we may assume that its company backing is good.

\paragraph{Documentation}

\begin{itemize}
\item Currentness: \textbf{good} - It seems like they have well divided and updated the documentation for each version. Though, it has the same problem as Sitewhere : you have to go to the \href{https://github.com/kaaproject/kaa/blob/master/doc/}{repository's documentation folder}.
\item Adaptability: \textbf{excellent} - The documentation itself is very-well separated by role of the reader. You have everything: administrator, hardware programmer, contributor guides. Each of these offer detailed explanations, with plenty of screenshots and code snippets to help you get started.
\end{itemize}

The overall quality of the documentation is very good, and makes you feel you can easily start using the solution right away.

\paragraph{UI-UX}

\begin{itemize}
\item Server Management UI: \textbf{good} - From what can be seen in the documentation, the administration UI is simple but complete, with a not so modern look, but we can bet that will be heavily upgraded in version 1.0.
\item Sample applications: \textbf{excellent} - Kaa offers a plethora of \href{https://github.com/kaaproject/sample-apps}{sample apps} that will help you get started, be it with hardware integration or client (apps) integration.
\end{itemize}

The user experience is especially good thanks to the many sample applications, and the clearly displayed will the Kaa project has to be extremely compatible.

\paragraph{Security} 

\begin{itemize}
\item Statements: \textbf{good} - A clear concern for security has been expressed by the CTO of the company in a \href{https://www.kaaproject.org/lets-address-dark-side-iot-tackling-security-community/}{blog post} and is also mentioned in their \href{https://www.kaaproject.org/faq/}{FAQ} and \href{http://docs.kaaproject.org/display/KAA/Endpoint+registration#Endpointregistration-Registrationsecurity}{documentation}. According to these sources, data communication between the components is secured using with 2048bits-RSA encryption and 256bits-AES signing. Apparently, Kaa also ensures secure data storage by enforcing database-level encryption, as well as tenant-level encryption, which is very good practice. In the FAQ, it is also said that Kaa is fault-tolerant, since the data is automatically replicated across the nodes.
\item Audits: \textbf{non-existent} - No mention on the product website, neither found any through a quick search on the web.
\end{itemize}

A clear concern for security is expressed, and some of the measures taken in favor of it are clearly explained. However, no independent security audit of the solution has ever been executed.

\paragraph{Connectivity/Flexibility}

\paragraph{Note:} With the incoming 1.0 version, connectivity and flexibility is apparently about to skyrocket, since according to the \href{https://www.kaaproject.org/apply-for-early-access-to-kaa-1-0-banana-beach/}{Early Access application page}, the platform will turn to a SDK-less, technology independent paradigm, with Out-of-the-box MQTT support.

\begin{itemize}
\item Number of compatible hardware platforms: \textbf{8} - Kaa provides 8 endpoint SdKs for many popular hardware platforms : RaspberryPi, ESP8266, ...
\item Number of supported protocols: \textbf{1} - Devices must use Kaa's own protocol (described \href{https://github.com/kaaproject/kaa-rfcs/}{here}), based on MQTT/CoAP. For easy implementation, Kaa provides endpoint SdKs for the many different platforms.
\item Number of SDK implementations: \textbf{5} - Kaa provides 5 endpoint SdKs for all main OSes : Linux, Android, MacOS, Windows, and finally, a generic SdK written in Java.
\item Modularity: \textbf{good} - it is already possible to change some components of Kaa pretty easily, for example, MongoDB and Cassandra are both supported as NoSQL DBs, but it is claimed that for the 1.0 version "Every component of the platform can be customized or substituted with 3rd party software".
\end{itemize}

Kaa overall boasts a very good connectivity and flexibility, and it seems they will pursue in this direction.

\paragraph{Scalability}

\begin{itemize}
\item Third-Party scalability: \textbf{excellent} - Kaa offers a choice between two NoSQL DBS, MongoDB and Cassandra, and two SQL DBS, MariaDB and PostgreSQL, which are highly scalable.
\item Platform scalability: \textbf{excellent} - Kaa has been thought as a cluster of server nodes, and uses Apache ZooKeeper to coordinate services. Kaa also includes load-balancing capabilities. (Cf. its \href{https://kaaproject.github.io/kaa/docs/v0.10.0/Architecture-overview/}{Architecture Overview})
\end{itemize}

Kaa seems to have been truly been built from the ground-up to be fully distributed.

\paragraph{Arduino/NodeMCU compatibility}

\begin{itemize}
\item Library: \textbf{good} - As said before, there is a dedicated SdK for the ESP8266, but not for the NodeMCU especially.
\item Boilerplate code: \textbf{good} - There is quite a little boilerplate C code, but it is well-explained, and it shouldn't be really too hard to work with it.
\end{itemize}

Relatively good compatibility with the ESP8266, however, the support for Arduino or NodeMCU is lacking.

\paragraph{Java} 

\begin{itemize}
\item Server percentage: \textbf{69.9\%} - Mainly developped in Java (Netty+Spring), almost all of its third party services are written in Java too.
\end{itemize}

\subsection{Device Hive}

\paragraph{Project Health}

\begin{itemize}
\item First Release Date: 
\item Latest Release Date: \textbf{2017/06/19} - 
\item Latest Commit Date: \textbf{2017/06/20} - 
\item Number of main contributors: \textbf{1} - 
\item Open issues ratio: \textbf{0.0534653465} - 
\end{itemize}

\paragraph{Company backing}

\begin{itemize}
\item Company Support: \textbf{excellent} - 
\item Company Adoption: \textbf{good} - 
\end{itemize}

\paragraph{Documentation}

\begin{itemize}
\item Currentness: \textbf{good} - 
\item Adaptability: \textbf{excellent} - 
\end{itemize}

\paragraph{UI-UX}

\begin{itemize}
\item Server Management UI: \textbf{good} - 
\item Sample applications: \textbf{good} - 
\end{itemize}

\paragraph{Security} 

\begin{itemize}
\item Statements: \textbf{average} - 
\item Audits: \textbf{non-existent} - 
\end{itemize}

\paragraph{Connectivity/Flexibility}

\begin{itemize}
\item Number of compatible hardware platforms: \textbf{4} - 
\item Number of supported protocols: \textbf{8} - 
\item Number of SDK implementations: \textbf{2} - 
\item Modularity: \textbf{good} - 
\end{itemize}

\paragraph{Arduino/NodeMCU compatibility}

\begin{itemize}
\item Library: \textbf{good} - 
\item Boilerplate code: \textbf{good} - 
\end{itemize}

\paragraph{Java} 

\begin{itemize}
\item Server percentage: \textbf{81.8\%} - 
\end{itemize}

\subsection{Zetta JS}

\paragraph{Project Health}

\begin{itemize}
\item First Release Date: 
\item Latest Release Date: \textbf{2017/06/19} - 
\item Latest Commit Date: \textbf{2017/06/20} - 
\item Number of main contributors: \textbf{1} - 
\item Open issues ratio: \textbf{0.0534653465} - 
\end{itemize}

\paragraph{Company backing}

\begin{itemize}
\item Company Support: \textbf{excellent} - 
\item Company Adoption: \textbf{good} - 
\end{itemize}

\paragraph{Documentation}

\begin{itemize}
\item Currentness: \textbf{good} - 
\item Adaptability: \textbf{excellent} - 
\end{itemize}

\paragraph{UI-UX}

\begin{itemize}
\item Server Management UI: \textbf{good} - 
\item Sample applications: \textbf{good} - 
\end{itemize}

\paragraph{Security} 

\begin{itemize}
\item Statements: \textbf{average} - 
\item Audits: \textbf{non-existent} - 
\end{itemize}

\paragraph{Connectivity/Flexibility}

\begin{itemize}
\item Number of compatible hardware platforms: \textbf{4} - 
\item Number of supported protocols: \textbf{8} - 
\item Number of SDK implementations: \textbf{2} - 
\item Modularity: \textbf{good} - 
\end{itemize}

\paragraph{Arduino/NodeMCU compatibility}

\begin{itemize}
\item Library: \textbf{good} - 
\item Boilerplate code: \textbf{good} - 
\end{itemize}

\paragraph{Java} 

\begin{itemize}
\item Server percentage: \textbf{81.8\%} - 
\end{itemize}

\subsection{ThingSpeak}

\paragraph{Project Health}

\begin{itemize}
\item First Release Date: 
\item Latest Release Date: \textbf{2017/06/19} - 
\item Latest Commit Date: \textbf{2017/06/20} - 
\item Number of main contributors: \textbf{1} - 
\item Open issues ratio: \textbf{0.0534653465} - 
\end{itemize}

\paragraph{Company backing}

\begin{itemize}
\item Company Support: \textbf{excellent} - 
\item Company Adoption: \textbf{good} - 
\end{itemize}

\paragraph{Documentation}

\begin{itemize}
\item Currentness: \textbf{good} - 
\item Adaptability: \textbf{excellent} - 
\end{itemize}

\paragraph{UI-UX}

\begin{itemize}
\item Server Management UI: \textbf{good} - 
\item Sample applications: \textbf{good} - 
\end{itemize}

\paragraph{Security} 

\begin{itemize}
\item Statements: \textbf{average} - 
\item Audits: \textbf{non-existent} - 
\end{itemize}

\paragraph{Connectivity/Flexibility}

\begin{itemize}
\item Number of compatible hardware platforms: \textbf{4} - 
\item Number of supported protocols: \textbf{8} - 
\item Number of SDK implementations: \textbf{2} - 
\item Modularity: \textbf{good} - 
\end{itemize}

\paragraph{Arduino/NodeMCU compatibility}

\begin{itemize}
\item Library: \textbf{good} - 
\item Boilerplate code: \textbf{good} - 
\end{itemize}

\paragraph{Java} 

\begin{itemize}
\item Server percentage: \textbf{81.8\%} - 
\end{itemize}

\subsection{Parse}

\paragraph{Project Health}

\begin{itemize}
\item First Release Date: 
\item Latest Release Date: \textbf{2017/06/19} - 
\item Latest Commit Date: \textbf{2017/06/20} - 
\item Number of main contributors: \textbf{1} - 
\item Open issues ratio: \textbf{0.0534653465} - 
\end{itemize}

\paragraph{Company backing}

\begin{itemize}
\item Company Support: \textbf{excellent} - 
\item Company Adoption: \textbf{good} - 
\end{itemize}

\paragraph{Documentation}

\begin{itemize}
\item Currentness: \textbf{good} - 
\item Adaptability: \textbf{excellent} - 
\end{itemize}

\paragraph{UI-UX}

\begin{itemize}
\item Server Management UI: \textbf{good} - 
\item Sample applications: \textbf{good} - 
\end{itemize}

\paragraph{Security} 

\begin{itemize}
\item Statements: \textbf{average} - 
\item Audits: \textbf{non-existent} - 
\end{itemize}

\paragraph{Connectivity/Flexibility}

\begin{itemize}
\item Number of compatible hardware platforms: \textbf{4} - 
\item Number of supported protocols: \textbf{8} - 
\item Number of SDK implementations: \textbf{2} - 
\item Modularity: \textbf{good} - 
\end{itemize}

\paragraph{Arduino/NodeMCU compatibility}

\begin{itemize}
\item Library: \textbf{good} - 
\item Boilerplate code: \textbf{good} - 
\end{itemize}

\paragraph{Java} 

\begin{itemize}
\item Server percentage: \textbf{81.8\%} - 
\end{itemize}

\subsection{DSA}

\paragraph{Project Health}

\begin{itemize}
\item First Release Date: 
\item Latest Release Date: \textbf{2017/06/19} - 
\item Latest Commit Date: \textbf{2017/06/20} - 
\item Number of main contributors: \textbf{1} - 
\item Open issues ratio: \textbf{0.0534653465} - 
\end{itemize}

\paragraph{Company backing}

\begin{itemize}
\item Company Support: \textbf{excellent} - 
\item Company Adoption: \textbf{good} - 
\end{itemize}

\paragraph{Documentation}

\begin{itemize}
\item Currentness: \textbf{good} - 
\item Adaptability: \textbf{excellent} - 
\end{itemize}

\paragraph{UI-UX}

\begin{itemize}
\item Server Management UI: \textbf{good} - 
\item Sample applications: \textbf{good} - 
\end{itemize}

\paragraph{Security} 

\begin{itemize}
\item Statements: \textbf{average} - 
\item Audits: \textbf{non-existent} - 
\end{itemize}

\paragraph{Connectivity/Flexibility}

\begin{itemize}
\item Number of compatible hardware platforms: \textbf{4} - 
\item Number of supported protocols: \textbf{8} - 
\item Number of SDK implementations: \textbf{2} - 
\item Modularity: \textbf{good} - 
\end{itemize}

\paragraph{Arduino/NodeMCU compatibility}

\begin{itemize}
\item Library: \textbf{good} - 
\item Boilerplate code: \textbf{good} - 
\end{itemize}

\paragraph{Java} 

\begin{itemize}
\item Server percentage: \textbf{81.8\%} - 
\end{itemize}

\subsection{Blynk}

\paragraph{Project Health}

\begin{itemize}
\item First Release Date: 
\item Latest Release Date: \textbf{2017/06/19} - 
\item Latest Commit Date: \textbf{2017/06/20} - 
\item Number of main contributors: \textbf{1} - 
\item Open issues ratio: \textbf{0.0534653465} - 
\end{itemize}

\paragraph{Company backing}

\begin{itemize}
\item Company Support: \textbf{excellent} - 
\item Company Adoption: \textbf{good} - 
\end{itemize}

\paragraph{Documentation}

\begin{itemize}
\item Currentness: \textbf{good} - 
\item Adaptability: \textbf{excellent} - 
\end{itemize}

\paragraph{UI-UX}

\begin{itemize}
\item Server Management UI: \textbf{good} - 
\item Sample applications: \textbf{good} - 
\end{itemize}

\paragraph{Security} 

\begin{itemize}
\item Statements: \textbf{average} - 
\item Audits: \textbf{non-existent} - 
\end{itemize}

\paragraph{Connectivity/Flexibility}

\begin{itemize}
\item Number of compatible hardware platforms: \textbf{4} - 
\item Number of supported protocols: \textbf{8} - 
\item Number of SDK implementations: \textbf{2} - 
\item Modularity: \textbf{good} - 
\end{itemize}

\paragraph{Arduino/NodeMCU compatibility}

\begin{itemize}
\item Library: \textbf{good} - 
\item Boilerplate code: \textbf{good} - 
\end{itemize}

\paragraph{Java} 

\begin{itemize}
\item Server percentage: \textbf{81.8\%} - 
\end{itemize}

\subsection{Paho}

\paragraph{Project Health}

\begin{itemize}
\item First Release Date: 
\item Latest Release Date: \textbf{2017/06/19} - 
\item Latest Commit Date: \textbf{2017/06/20} - 
\item Number of main contributors: \textbf{1} - 
\item Open issues ratio: \textbf{0.0534653465} - 
\end{itemize}

\paragraph{Company backing}

\begin{itemize}
\item Company Support: \textbf{excellent} - 
\item Company Adoption: \textbf{good} - 
\end{itemize}

\paragraph{Documentation}

\begin{itemize}
\item Currentness: \textbf{good} - 
\item Adaptability: \textbf{excellent} - 
\end{itemize}

\paragraph{UI-UX}

\begin{itemize}
\item Server Management UI: \textbf{good} - 
\item Sample applications: \textbf{good} - 
\end{itemize}

\paragraph{Security} 

\begin{itemize}
\item Statements: \textbf{average} - 
\item Audits: \textbf{non-existent} - 
\end{itemize}

\paragraph{Connectivity/Flexibility}

\begin{itemize}
\item Number of compatible hardware platforms: \textbf{4} - 
\item Number of supported protocols: \textbf{8} - 
\item Number of SDK implementations: \textbf{2} - 
\item Modularity: \textbf{good} - 
\end{itemize}

\paragraph{Arduino/NodeMCU compatibility}

\begin{itemize}
\item Library: \textbf{good} - 
\item Boilerplate code: \textbf{good} - 
\end{itemize}

\paragraph{Java} 

\begin{itemize}
\item Server percentage: \textbf{81.8\%} - 
\end{itemize}

\subsection{Kura}

\paragraph{Project Health}

\begin{itemize}
\item First Release Date: 
\item Latest Release Date: \textbf{2017/06/19} - 
\item Latest Commit Date: \textbf{2017/06/20} - 
\item Number of main contributors: \textbf{1} - 
\item Open issues ratio: \textbf{0.0534653465} - 
\end{itemize}

\paragraph{Company backing}

\begin{itemize}
\item Company Support: \textbf{excellent} - 
\item Company Adoption: \textbf{good} - 
\end{itemize}

\paragraph{Documentation}

\begin{itemize}
\item Currentness: \textbf{good} - 
\item Adaptability: \textbf{excellent} - 
\end{itemize}

\paragraph{UI-UX}

\begin{itemize}
\item Server Management UI: \textbf{good} - 
\item Sample applications: \textbf{good} - 
\end{itemize}

\paragraph{Security} 

\begin{itemize}
\item Statements: \textbf{average} - 
\item Audits: \textbf{non-existent} - 
\end{itemize}

\paragraph{Connectivity/Flexibility}

\begin{itemize}
\item Number of compatible hardware platforms: \textbf{4} - 
\item Number of supported protocols: \textbf{8} - 
\item Number of SDK implementations: \textbf{2} - 
\item Modularity: \textbf{good} - 
\end{itemize}

\paragraph{Arduino/NodeMCU compatibility}

\begin{itemize}
\item Library: \textbf{good} - 
\item Boilerplate code: \textbf{good} - 
\end{itemize}

\paragraph{Java} 

\begin{itemize}
\item Server percentage: \textbf{81.8\%} - 
\end{itemize}

\subsection{Kapua}

\paragraph{Project Health}

\begin{itemize}
\item First Release Date: 
\item Latest Release Date: \textbf{2017/06/19} - 
\item Latest Commit Date: \textbf{2017/06/20} - 
\item Number of main contributors: \textbf{1} - 
\item Open issues ratio: \textbf{0.0534653465} - 
\end{itemize}

\paragraph{Company backing}

\begin{itemize}
\item Company Support: \textbf{excellent} - 
\item Company Adoption: \textbf{good} - 
\end{itemize}

\paragraph{Documentation}

\begin{itemize}
\item Currentness: \textbf{good} - 
\item Adaptability: \textbf{excellent} - 
\end{itemize}

\paragraph{UI-UX}

\begin{itemize}
\item Server Management UI: \textbf{good} - 
\item Sample applications: \textbf{good} - 
\end{itemize}

\paragraph{Security} 

\begin{itemize}
\item Statements: \textbf{average} - 
\item Audits: \textbf{non-existent} - 
\end{itemize}

\paragraph{Connectivity/Flexibility}

\begin{itemize}
\item Number of compatible hardware platforms: \textbf{4} - 
\item Number of supported protocols: \textbf{8} - 
\item Number of SDK implementations: \textbf{2} - 
\item Modularity: \textbf{good} - 
\end{itemize}

\paragraph{Arduino/NodeMCU compatibility}

\begin{itemize}
\item Library: \textbf{good} - 
\item Boilerplate code: \textbf{good} - 
\end{itemize}

\paragraph{Java} 

\begin{itemize}
\item Server percentage: \textbf{81.8\%} - 
\end{itemize}

\subsection{Hono}

\paragraph{Project Health}

\begin{itemize}
\item First Release Date: 
\item Latest Release Date: \textbf{2017/06/19} - 
\item Latest Commit Date: \textbf{2017/06/20} - 
\item Number of main contributors: \textbf{1} - 
\item Open issues ratio: \textbf{0.0534653465} - 
\end{itemize}

\paragraph{Company backing}

\begin{itemize}
\item Company Support: \textbf{excellent} - 
\item Company Adoption: \textbf{good} - 
\end{itemize}

\paragraph{Documentation}

\begin{itemize}
\item Currentness: \textbf{good} - 
\item Adaptability: \textbf{excellent} - 
\end{itemize}

\paragraph{UI-UX}

\begin{itemize}
\item Server Management UI: \textbf{good} - 
\item Sample applications: \textbf{good} - 
\end{itemize}

\paragraph{Security} 

\begin{itemize}
\item Statements: \textbf{average} - 
\item Audits: \textbf{non-existent} - 
\end{itemize}

\paragraph{Connectivity/Flexibility}

\begin{itemize}
\item Number of compatible hardware platforms: \textbf{4} - 
\item Number of supported protocols: \textbf{8} - 
\item Number of SDK implementations: \textbf{2} - 
\item Modularity: \textbf{good} - 
\end{itemize}

\paragraph{Arduino/NodeMCU compatibility}

\begin{itemize}
\item Library: \textbf{good} - 
\item Boilerplate code: \textbf{good} - 
\end{itemize}

\paragraph{Java} 

\begin{itemize}
\item Server percentage: \textbf{81.8\%} - 
\end{itemize}

\section{Summarized analysis}

\section{Conclusion}

\pagebreak
\begin{titlepage}
    \vspace*{\fill}
        \begin{center}
          \textbf{\huge - Annexes -}\\[20pt]
        \end{center}
    \vspace*{\fill}
\end{titlepage}

\end{document}
